\documentclass[UTF8, twocolumn ]{ctexart}
\usepackage{graphicx}
\usepackage{amsmath}
\usepackage{amssymb}
\usepackage{paralist}
\usepackage[
  colorlinks,
  linkcolor = black
]{hyperref}

\usepackage{fancyhdr}                                
\usepackage{lastpage}                                           
\usepackage{layout}                                                                          

\linespread{1.56}
\columnsep = 15pt
\begin{document}

\title{\huge{基于正态分布概率计算和支持向量机计算的WiFi定位技术}}
%\author{北京优锐科技有限公司\ 丁贵金\ 朱韬\ 袁万尚}
\author{北京优锐科技有限公司\ 朱韬}
\date{\today\\*\ \hrule}
\maketitle

%%%%%%%%%%%%%%%%%%%%%%%%%%%%%%%%%%%%%%%%%%%%%%%

\begin{abstract}
  基于概率和支持向量机原理的 WiFi定位技术,不需要依赖专用设备,部署简单使用便捷,对环境无强制依赖,可以在复杂 WiFi 环境下实现移动设备精确定位。该 WiFi 定位技术的核心原理是支持向量机,辅助正态分布的概率计算来优化支持向量机计算过程。
  \par
  本文针对的设备:是带有WiFi功能的移动设备。针对的环境:是分布着大量WiFi接入设备的室内环境。实现的主要目标:是通过WiFi移动设备,在分布着大量WiFi接入设备的室内环境中,实现精确定位。
  \par
  \noindent{\textbf{关键词}:}
  \par
  WiFi定位,室内定位,支持向量机。
\end{abstract}

%%%%%%%%%%%%%%%%%%%%%%%%%%%%%%%%%%%%%%%%%%%%%%%

\section{引言}
本文的“名词、概念和常识说明”部分,主要解释全文中出现的各种专用名词,涉及到的所有概念,以及阅读本文所需的技术常识。“实现方法”部分则具体阐述技术实现过程,“讨论”部分主要分析了实现过程中可能出现的各种问题,以及处理问题的方法。“技术创新”部分具体说明该技术的先进性,与同类技术相比下的优势,“应用前景”部分介绍了该技术实际应用的具体形式,以及对采用该技术的行业所产生的积极作用。

%%%%%%%%%%%%%%%%%%%%%%%%%%%%%%%%%%%%%%%%%%%%%%%

\section{名词、概念和常识说明}
\subsection{定位空间(CR)}
定位空间(calibration room),指的是提供定位功能的空间。例如采用WiFi定位的公司内部空间,公司外部无限大的空间就是不可定位空间。如图:
\subsection{移动终端设备}
本文中的“移动终端设备”,是指拥有WiFi连接能力的移动设备,如有WiFi功能的智能手机、平板电脑和其他便携的移动终端设备。
\subsection{WiFi接入设备(AP)}
提供WiFi接入服务的硬件设备,如无线路由器。本文中对WiFi接入设备的要求,是必须能够被移动终端设备识别到的WiFi接入设备。
\subsection{AP名(MAC)}
AP名就是AP的MAC地址,是WiFi接入设备的唯一标示。
\subsection{AP信号强度(RSS)}
AP信号强度,是移动终端设备对WiFi接入设备信号强度的识别结果。该结果是一个从0到-100的值,该值是一个指数值,代表信号的强度指数,并不是真正的物理信号量。不同的移动终端设备,对同一个WiFi接入设备,信号强度值的识别结果是不同的。
\subsection{AP信号强度平均值(mRSS)}
同一个移动终端设备,对同一个WiFi接入设备,每次信号强度的识别结果也是不同的。一个移动终端设备,对同一个WiFi接入设备,进行多次采集,识别到一组信号强度,对该组信号强度的求平均值,就得到了信号强度平均值,计算公式如下:
,N为该组信号强度的个数。
\subsection{WiFi接入设备数据(APD)}
WiFi接入设备数据包括,WiFi接入设备名(MAC)和WiFi接入设备被识别到的信号强度。
\subsection{待定位点(LP)}
带定位点,是指一个移动终端设备,出现在定位空间内,该移动终端设备自身的位置。
\subsection{定位区域(value)}
定位空间是一个完整的封闭空间,要在这个空间内实现WiFi定位,就必须将这个完整的空间划分为多个小区域,每个区域都有一个ID。如图:

\subsection{定位校准点(CP)}
在定位区域中,指定一个位置点,这个点就是定位校准点。如图:

\subsection{定位区域数据}
在定位区域内的定位校准点上,采集到的WiFi接入设备数据,就是定位区域采集数据。该数据的标示是区域ID,内容是WiFi接入设备数据。如图:

\subsection{校准数据}
校准数据,是对所有定位区域数据的整理计算结果。如图:

Value是定位区域ID,AP是收集到的所有WiFi接入设备名,mRSS是该WiFi接入设备被识别到的,信号强度平均值。
\subsection{待定位数据}
移动设备在待定位点,某一时刻采集到的WiFi接入设备数据。如图:

AP是识别到的所有WiFi接入设备名,RSS是识别到的所有WiFi设备的信号强度。
\subsection{概率}
\subsubsection{正态分布}
正态分布是一种常用的概率分布函数,本文阐述的WiFi定位技术,在实现过程中采用了这种数学方法。公式如下:

\subsubsection{概率值(Pr)}
本文中所讲的概率值,是指一个待定位点,出现在某一定位区域中的概率。
\subsection{支持向量机(SVM)}
支持向量机,是一种机器学习原理,是采用数学方法,实现对某种向量值进行特定分类的工具。
\subsubsection{样本}
本文所说的样本,是指WiFi接入设备的数据。
\subsubsection{样本空间}
多个样本生成的集合,叫做样本空间。
特征向量
本文所说的特征向量,是指可以代表一个定位区域的样本。是校准数据中,一个定位区域中所有样本的子集。
\subsubsection{核函数}
核函数,是指支持向量机所采用的某种特征向量训练模式。本文中的核函数采用RBF核函数。
\subsubsection{惩罚因子}
惩罚因子,是指在支持向量机计算中,分割两个样本空间时,对不可分样本的包容程度。


%%%%%%%%%%%%%%%%%%%%%%%%%%%%%%%%%%%%%%%%%%%%%%%

\section{实现方法}
\subsection{总体流程}
\subsection{确定CR、Value、CP}
\subsection{收集各个Value上的定位区域数据}
\subsection{生成校准数据}
\subsection{得到移动终端设备的待定位数据}
\subsection{计算LP出现在各个Value上的概率,得到概率最大的两个Value}
\subsection{将得到的两个Value对应的校准数据输入SVM进行训练}
\subsection{将待定位数据输入SVM进行计算,得到最终定位结果}

%%%%%%%%%%%%%%%%%%%%%%%%%%%%%%%%%%%%%%%%%%%%%%

\section{讨论}


%%%%%%%%%%%%%%%%%%%%%%%%%%%%%%%%%%%%%%%%%%%%%%

\section{技术创新(权利保护)}

\section{应用前景}
WiFi定位技术可以应用在多种领域,主要为行业提供室内定位基础服务。本文阐述的WiFi定位技术,不依赖特定硬件和特殊环境要求,部署应用方便灵活,成本低,可以结合多种具体的业务实现垂直服务。
\subsection{室内定位和导航}
室内定位和导航,是WiFi最主要的应用方式,也是最直接的服务提供模式。
\begin{enumerate}
\item 公共室内空间的位置服务,如商场、超市和大规模的综合购物中心。为了方便顾客找到所需商品位置,或快速找到某个区域(例如卫生间、出口、餐饮区,等等),结合WiFi定位和电子地图技术,实现顾客自身位置确定和导航服务功能。
\item 室内区域流量统计服务,如在商场、超市和大规模的综合购物中心,有时业主需要统计一段时间内,某个区域或所有区域的客流信息,或者某个客流最大的热点区域。结合WiFi定位和电子地图以及大数据技术,可以实现室内区域的流量统计,顾客使用移动设备上的WiFi定位客户端软件,经过的路径会推送至服务器,服务器端根据统计整理,最后分析出某个或所有区域的客流数据。
\end{enumerate}
\subsection{地理围栏}
地理围栏是一个新兴的移动互联网服务概念,其主要目的是实现互联网应用和移动设备位置信息的结合。
\begin{enumerate}
\item 设备位置开启移动应用。当移动设备进入某个特定区域后,便自动开启一些应用软件。这种服务适合应用在商场导购,医院就医指导,机场火车站等公共服务场所。
\item 特定移动应用的使用范围。移动设备上的某些软件,只能在某些特定的环境中才可以使用。这种服务适合公司企业,用来实现现代移动化的企业管理和自动化办公。当员工进入企业办公区域后,相关工作的信息服务和业务软件才可以工作,保证了企业数据的安全,也将企业管理简单有效的实现。
\end{enumerate}
\subsubsection{移动设备防盗}
某些非个人使用的移动设备,如餐厅的点餐设备,高级场馆的自助服务设备,都是为了本地服务存在的。为了防止个别人将其私自带出,又要实现该设备的自由使用,就必须采用一种防盗的技术做应用保证。本文阐述的WiFi定位技术,可以划出移动设备的安全范围,当设备离开这个安全区域后,便可以通过报警、监控和追踪等方式及时发现并处理。
\subsubsection{区域信息安全}
信息隔离,是信息安全中的一项重要内容,指信息内容与信息的有效区域之间的对应关系。采用本文阐述的WiFi定位技术,可以通过软件安全策略,规定信息的安全范围,当移动设备需要读取某些带有安全级别的数据时,会根据移动设备所处的位置进行操作合法性判断,当发现移动设备所处的位置不合法时,便会组织数据的读取。对于已经读取到的数据,当设备离开该数据的安全区域后,通过软件安全策略谁自动删除该数据,保证数据存在的有效范围与数据的安全范围一致。
\subsubsection{地理信息隔离}
移动设备的最大特点是可以随身携带,可以随时随地的产生数据,如照相、笔记、下载或应用软件生成数据。利用本文所阐述的WiFi定位技术,可以在产生数据的同时,将数据与其产生时的位置做关联。无论在以后的搜索查找,或是统计,位置信息会提供跟多的有效途径,版主使用者对数据的管理。也可以将数据的存储或显示方式,与位置信息关联,是某些数据只在特定位置区域是才可以被操作。
\subsection{游戏}
电子游戏的场景通常是虚拟的空间,但是随着技术的发展,电子游戏的实现也可以利用现实空间。本文阐述的WiFi定位技术,可以实现真实空间在设备上的体现,电子游戏的方式也就从单一的虚拟空间扩展到了真实空间,游戏玩家除了操作设备,还需要亲身行动来完成游戏。这样的组合不仅扩展了电子游戏的功能,甚至可以改变电子游戏的传统运行模式。
%%%%%%%%%%%%%%%%%%%%%%%%%%%%%%%%%%%%%%%%%%%%%%

\end{document}
